\subsection{Properties and Attributes}
This section describes translation rules for the metadata set on repository files and directories in Git or Subversion.
Aim of Translator is to preserve meaning of the metadata information while translating from one version control system
to another.
\\\\
Subversion keeps metadata in form of versioned properties that could be set on any file or directory in the repository.
Properties might have any name, but there are number of reserved names, with \emph{svn:} prefix for special properties. 
These special properties, been set on a file or directory, controls various aspects of Subversion client behavior, 
from EOL substitution to support of the external repositories.
\\\\
Git has multiple ways to keep metadata, these ways are:
\begin{enumerate}
\compactlist
\item Every directory and file has \emph{file mode} bytes stored at repository;
\item \emph{.gitignore} file stores file path patterns to be ignored by Git operations;
\item \emph{.gitattributes} is a file in a special format that could be present in any directory.
This file allows to define special metadata attributes per file.
\end{enumerate}

Special files stored in Git repository, in particular .gitignore and .gitattributes are kept
on content translation of Git commit to Subversion revision and thus contents of these files is not 
accessible to Subversion users. It is not allowed to created new files and directories named
.gitignore or .gitattributes in the Subversion repository.
\\\\
Translation rules for the metadata properties and attributes are described in further sections.

\subsubsection{End-of-Line Bytes and MIME Types}
Subversion supports EOL bytes substitution since its early versions. 
Support of different EOL-styles was introduced in Git at version 1.7.2. Taking this in account, 
Translator will perform valid EOL translation only for the users of Git version 1.7.2 or newer.
\\\\
EOL substitution only makes sense for text files only. Thus, in this specification, rules 
for translating EOL-style and MIME-type metadata are combined. 
\\\\
\textbf{Subversion to Git Translation}
\\\\
Subversion uses svn:eol-style and svn:mime-type properties set on individual files to keep corresponding kinds of metadata.
Value of these properties set on a file are translated into the corresponding lines in \emph{.gitattributes} file, which ensures that 
corresponding file in the Git repository has proper \emph{eol} and \emph{text} attributes set.
\\\\
Upon translation of these properties Translator tries to minimize amount of modifications
made to .gitattributes files. To translate svn:eol-style and svn:mime-type properties changes, 
Translator uses the following algorithm:

\begin{enumerate}
\compactlist
\item For every file which has svn:eol-style or svn:mime-type property modified, Translator looks for a corresponding line in one of .gitattributes files located in the parent directory or 
at the higher level of the directories hierarchy.\\
	
\item If attribute line is found, Translator adjusts corresponding attribute to reflect performed change. Basically Translator adds a line with pattern corresponding to the path of modified file and a new attribute:\\\\
\emph{/$<$relative-file-path$>$ $<$text-related-value$>$ $<$eol-related-value$>$},\\\\
where $<$text-related-value$>$ and $<$eol-related-value$>$ are generated according to the rules defined in the table \ref{eol_mime_svn_to_git}.\\
	
\item If no .gitattributes file was found in the ancestor directories of the file, Translator creates new .gitattributes file in the parent directory of the file with the following contents:\\\\
\emph{/$<$file-name$>$ $<$text-related-value$>$ $<$eol-related-value$>$},\\\\
where $<$text-related-value$>$ and $<$eol-related-value$>$ are generated according to the rules defined in the table \ref{eol_mime_svn_to_git}.\\
	
\item If more than one .gitattributes file was found in the parent directories of the file, then the closest one is chosen. 
Translator appends the following line to the selected .gitattributes file:\\\\
\emph{/$<$relative-file-path$>$ $<$text-related-value$>$ $<$eol-related-value$>$},\\\\
where $<$text-related-value$>$ and $<$eol-related-value$>$ are generated according to the rules defined in the table \ref{eol_mime_svn_to_git}.\\
\item Finally, for each created or modified .gitattributes file, Translator compacts added rules transforming 
explicit path patterns to the recursive path patterns where possible.
\end{enumerate}

\emph{Note}: comments stored in .gitattributes file left unmodified by Translator.\\
\emph{Note}: Git user must set \emph{core.eol} configuration parameter to allow correct handling of the \emph{native} value of svn:eol-style property.

\begin{center}
\begin{tabular}{ | p{0.20\textwidth} | p{0.20\textwidth} | p{0.20\textwidth} | p{0.20\textwidth} |}
	\hline
	svn:eol-style &   svn:mime-type &   text  & eol \\ \hline \hline
	-             &   text/-        &   unset & undef \footnotemark[1] \footnotemark[2] \\ \hline
	-             &   binary        &   unset & undef \\ \hline
	lf            &   text/-        &   set   & lf \\ \hline
	lf            &   binary        &   unset & undef \\ \hline
	cr            &   text/-        &   unset & undef  \footnotemark[3] \\ \hline
	cr            &   binary        &   unset & undef \\ \hline
	crlf          &   text/-        &   set   & crlf \\ \hline
	crlf          &   binary        &   unset & undef \\ \hline
	native        &   text/-        &   set   & undef \footnotemark[4] \\ \hline
	native        &   binary        &   unset & undef \\ \hline
\end{tabular}
\captionof{table}{Subversion to Git EOL and MIME translation.}
\label{eol_mime_svn_to_git}
\footnotetext[1]{\emph{undef}, in general, needs only to be set, if there is another rule which matches the file, otherwise we may skip these attributes at all.}
\footnotetext[2]{Missing Subversion properties mean text, leave EOLs as is for that case.}
\footnotetext[3]{Subversion recognizes CR EOL-style, Git does not.}
\footnotetext[4]{svn:eol-style property value \emph{native} should stay the same, Git user may configure \emph{native} via \emph{core.eol} configuration property.} 
\end{center}
\newpage
\textbf{Git to Subversion Translation}\\

When .gitattributes file is modified in Git commit, Translator uses the following algorithm steps to apply changes to the metadata of corresponding files in Subversion repository:

\begin{enumerate}
\compactlist
\item For every file affected by the change new svn:eol-style and svn:mime-type properties are computed accordingly to the rules defined in the table \ref{eol_mime_git_to_svn}.
\item When there are no changes detected, Translator performs no further actions.
\item When new and old properties values differ only in those values which were computed from \emph{undef} attribute value, Translator performs no further actions.
\item Otherwise Translator sets new properties to affected files.
\end{enumerate}

\begin{center}
\begin{tabular}{ | p{0.20\textwidth} | p{0.20\textwidth} | p{0.20\textwidth} | p{0.20\textwidth} |}
	\hline
	text  & eol      &  svn:eol-style  &  svn:mime-type \\ \hline \hline
	unset & undef    &  -              &  -/binary \footnotemark[1] \\ \hline
	unset & unset    &  -              &  -/binary \footnotemark[2] \\ \hline
	unset & set      &  native         &  - \footnotemark[3] \\ \hline
	unset & lf       &  lf             &  - \\ \hline
	unset & crlf     &  crlf           &  - \\ \hline
	set   & undef    &  native         &  - \\ \hline
	set   & unset    &  native         &  - \footnotemark[4] \\ \hline
	set   & set      &  native         &  - \\ \hline
	set   & lf       &  lf             &  - \\ \hline
	set   & crlf     &  crlf           &  - \\ \hline
	auto  & undef    &  native/-       &  -/binary \footnotemark[1] \\ \hline
	auto  & unset    &  native/-       &  -/binary \footnotemark[5] \\ \hline
	auto  & set      &  native/-       &  -/binary \\ \hline
	auto  & lf       &  lf/-           &  -/binary \\ \hline
	auto  & crlf     &  crlf/-         &  -/binary \\ \hline
\end{tabular}
\captionof{table}{Git to Subversion EOL and MIME translation.}
\label{eol_mime_git_to_svn}
\footnotetext[1]{Perform binary check.}
\footnotetext[2]{unset/unset is redundant, treat it like unset/undef.}
\footnotetext[3]{From Git documentation:\\eol - This attribute sets a specific line-ending style to be used in the working directory. It enables end-of-line normalization without any content checks, effectively setting the text attribute.}
\footnotetext[4]{set/unset is contradiction, but text has precedence over eol, hence treat like set/undef.}
\footnotetext[5]{auto/unset is contradiction, but text has precedence over eol, hence treat like auto/undef}
\end{center}

\subsubsection{Symbolic Links}
Git stores symbolic link as an entry at Tree Object with a special file mode and corresponding blob containing \emph{path/to/target}.
\\\\
Subversion represents symbolic link as a file with content \emph{link path/to/target} and a property svn:special set on it.
\\\\
So translation is performed by adding or removing \emph{link } prefix to the file content and setting the mode or the property.
\\\\
If file at Subversion repository has svn:special property but its content doesn't start with \emph{link } prefix, it is considered as an ordinary file and translated as a blob.

\subsubsection{Executables}
In Git repository, executable files are marked with a special file mode. Executable file in Subversion repository keeps svn:executable property. 
Thus, svn:executable property is translated into Git file mode and vice versa directly.

\subsubsection{Ignores}
Subversion uses svn:ignore property which may be set on a directory to define a list of
file name patterns. Files in this very directory that match at least one of the patterns are ignored.
\\\\
Git uses versioned .gitignore file to define file path patterns to be ignored. Files in the directory
where .gitignores file is located as well as files in the child directories are matched against defined patterns.
\\\\
\textbf{Subversion to Git Translation}
\\\\
Upon translation of the svn:ignore properties changes Translator tries to minimize amount of modifications
made to .gitignore files. To translate svn:ignore properties changes, Translator uses the following algorithm:

\begin{enumerate}
\compactlist
\item For every directory with modified svn:ignore property Translator looks for a corresponding line in one of the 
.gitignore files located in the parent directory or at the higher level of the directories hierarchy.\\
\item If ignore pattern line is found, Translator adjusts corresponding ignore rule to reflect performed change. 
Translator adds a line:\\\\
\emph{/$<$relative-file-path$>$/$<$ignore-pattern$>$}\\
if svn:ignore property was added or modified, and \\\\
\emph{!/$<$relative-file-path$>$/$<$ignore-pattern$>$}\\
if svn:ignore property was removed from the directory.\\
\item If no .gitignore file was found was found in the ancestor directories of the file, Translator creates a new .gitignore file in the file's directory with the following contents:\\\\
\emph{/$<$ignore-pattern$>$}\\
\item 
If more than one .gitignore file was found in the parent directories of the file, then the closest one is chosen. 
Translator appends the following line to the selected .gitignore file:\\\\
\emph{/$<$relative-file-path$>$/$<$ignore-pattern$>$}\\
\item Finally, for each created or modified .gitignore file, Translator compacts added rules transforming 
explicit path patterns to the recursive path patterns when possible.
\end{enumerate}
\emph{Note}: comments stored in .gitignore file left unmodified by Translator.\\
\\
\textbf{Git to Subversion Translation}
\\\\
When .gitignore file is modified in Git commit, Translator uses the following algorithm steps to apply changes to the svn:ignore properties of the corresponding directories in Subversion repository:

\begin{enumerate}
\compactlist
\item For every directory affected by the change, new svn:ignore property is computed according to the ignore patterns of all .gitignore files located at the directory and its parent directories.\\
\item When there is difference between newly computed and existing svn:ignore values detected, then Translator sets computed svn:ignore properties on affected directories.\\
\end{enumerate}
\subsubsection{Subversion Externals and Git Submodules}
\label{section_externals_and_submodules}
Git submodule is translated to the svn:external property with a fixed revision in case Git submodule refers to the Git repository
which is hosted on the GitHub.
\\\\
Git submodules referring to the third-party repositories are left untranslated.
\\\\
Subversion svn:external property is translated to the Git submodule in case the following conditions are met:
\begin{itemize}
\item svn:external points to the branch of Subversion repository hosted at GitHub;
\item svn:external refers to a fixed revision, not to the HEAD one.
\end{itemize}
Subversion external properties referring to third-party repositories, or to the non-branch directories of the GitHub Subversion repositories as
well as any file externals are left untranslated.