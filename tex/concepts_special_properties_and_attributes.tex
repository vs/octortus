\subsection{Properties and Attributes}
\subsubsection{End-of-Line Bytes and MIME Types}
Subversion supports EOL bytes substitution since its early versions. Support of different EOL-styles was introduced in Git at version 1.7.2. Translator is supposed to perform valid EOL translation for users of Git version 1.7.2 or above.
\\\\
EOL substitution makes sense for text files only, so Translator performs translation of both EOL-style and MIME-type metadata. Subversion uses svn:eol-style and svn:mime-type properties for this kind of metadata. Git introduces "eol" and "text" attributes for that purpose.
\\\\
Considering certain file at Subversion repository we can find its counterpart at Git repository. In both repositories file and its counterpart have corresponding metadata related to EOL style and MIME type. Modification of the metadata at one repository is translated to the metadata of the counterpart by Translator. 

\clearpage

For every file in Subversion repository Translator performs translation of its metadata to the metadata of corresponding file in Git repository. Translation rules are specified at table \ref{eol_mime_svn_to_git}.
\\\\\\\\\\\\

\begin{center}
\begin{tabular}{ | l | l | l | l |}
	\hline
	svn:eol-style &   svn:mime-type &   text  & eol \\ \hline \hline
	-             &   text/-        &   unset & undef \footnotemark[1] \footnotemark[2] \\ \hline
	-             &   binary        &   unset & undef \\ \hline
	lf            &   text/-        &   set   & lf \\ \hline
	lf            &   binary        &   unset & undef \\ \hline
	cr            &   text/-        &   unset & undef  \footnotemark[3] \\ \hline
	cr            &   binary        &   unset & undef \\ \hline
	crlf          &   text/-        &   set   & crlf \\ \hline
	crlf          &   binary        &   unset & undef \\ \hline
	native        &   text/-        &   set   & undef \footnotemark[4] \\ \hline
	native        &   binary        &   unset & undef \\ \hline
\end{tabular}
\captionof{table}{Subversion to Git EOL and MIME translation.}
\label{eol_mime_svn_to_git}
\footnotetext[1]{\emph{undef}, in general, needs only to be set, if there is another rule which matches the file, otherwise we may skip that attributes at all.}
\footnotetext[2]{Missing Subversion properties mean text, leave EOLs as is for that case.}
\footnotetext[3]{Subversion recognizes CR EOL-style, Git does not.}
\footnotetext[4]{svn:eol-style property value \emph{native} should stay the same, Git user may configure \emph{native} via \emph{core.eol} configuration property.} 
\end{center}

\clearpage

In opposite directory for every file in Git repository Translator performs translation of its metadata to the metadata of corresponding file in Subversion repository using the rules specified at table \ref{eol_mime_git_to_svn}.

\begin{center}
\begin{tabular}{ | l | l | l | l |}
	\hline
	text  & eol      &  svn:eol-style  &  svn:mime-type \\ \hline
	unset & undef    &  -              &  -/binary \footnotemark[1] \\ \hline
	unset & unset    &  -              &  -/binary \footnotemark[2] \\ \hline
	unset & set      &  native         &  - \footnotemark[3] \\ \hline
	unset & lf       &  lf             &  - \\ \hline
	unset & crlf     &  crlf           &  - \\ \hline
	set   & undef    &  native         &  - \\ \hline
	set   & unset    &  native         &  - \footnotemark[4] \\ \hline
	set   & set      &  native         &  - \\ \hline
	set   & lf       &  lf             &  - \\ \hline
	set   & crlf     &  crlf           &  - \\ \hline
	auto  & undef    &  native/-       &  -/binary \footnotemark[1] \\ \hline
	auto  & unset    &  native/-       &  -/binary \footnotemark[5] \\ \hline
	auto  & set      &  native/-       &  -/binary \\ \hline
	auto  & lf       &  lf/-           &  -/binary \\ \hline
	auto  & crlf     &  crlf/-         &  -/binary \\ \hline
\end{tabular}
\captionof{table}{Git to Subversion EOL and MIME translation.}
\label{eol_mime_git_to_svn}
\footnotetext[1]{Perform binary check.}
\footnotetext[2]{unset/unset is redundant, treat it like unset/undef.}
\footnotetext[3]{From Git documentation:\\eol - This attribute sets a specific line-ending style to be used in the working directory. It enables end-of-line normalization without any content checks, effectively setting the text attribute.}
\footnotetext[4]{set/unset is contradiction, but text has precedence over eol, hence treat like set/undef.}
\footnotetext[5]{auto/unset is contradiction, but text has precedence over eol, hence treat like auto/undef}
\end{center}

In Subversion EOL style and MIME type metadata stored as properties of the file, but in Git this metadata could be specified at any parent directory of the file in special .gitignore file. 

\subsubsection{Symbolic Links}
Git stores symbolic link as an entry at Tree Object with a special file mode and corresponding blob containing "path/to/target".
\\\\
Subversion represents symbolic link as a file with content "link path/to/target" and a property svn:special set on it.
\\\\
So translation is performed by adding or removing "link " prefix to the file content and setting the mode or the property.
\\\\
If file at Subversion repository has svn:special property but its content doesn't start with "link " prefix, it is considered as an ordinary file and translated as a blob.
\subsubsection{Ignores}
\subsubsection{Externals and Submodules}